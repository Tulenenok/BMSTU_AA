\chapter{Конструкторская часть}
В этом разделе будут приведены требования к вводу и программе, а также схемы алгоритмов нахождения расстояний Левенштейна и Дамерау-Левенштейна.

\section{Требования к вводу}
\begin{enumerate}
	\item На вход подаются две строки.
	\item Буквы верхнего и нижнего регистров считаются различными.
\end{enumerate}

\section{Требования к программе}
\begin{enumerate}
	\item Две пустые строки - корректный ввод, программа не должна аварийно завершаться.
	\item На выход программа должна вывести число - расстояние Левенштейна (Дамерау-Левенштейна).
\end{enumerate}

\section{Схема алгоритма нахождения расстояния Левенштейна}

На рисунке \ref{img:l} приведена схема нерекурсивного алгоритма нахождения расстояния Левенштейна.

\section{Схема алгоритма нахождения расстояния Дамерау — Левенштейна}

На рисунке \ref{img:dl} приведена схема нерекурсивного алгоритма нахождения расстояния Дамерау — Левенштейна.

На рисунке \ref{img:rdl} приведена схема рекурсивного алгоритма нахождения расстояния Дамерау — Левенштейна.

На рисунке \ref{img:rdl_cache} приведена схема рекурсивного алгоритма нахождения расстояния Дамерау — Левенштейна с использованием кеша в виде матрицы.

\section*{Вывод}

Перечислены требования к вводу и программе, а также на основе теоретических данных, полученных из аналитического раздела были построены схемы требуемых алгоритмов.

\img{200mm}{l}{Схема нерекурсивного алгоритма нахождения расстояния Левенштейна}

\img{200mm}{dl}{Схема нерекурсивного алгоритма нахождения расстояния Дамерау — Левенштейна}

\img{180mm}{rdl}{Схема рекурсивного алгоритма нахождения расстояния Дамерау — Левенштейна}

\img{200mm}{rdl_cache}{Схема рекурсивного алгоритма нахождения расстояния Дамерау — Левенштейна с использованием кеша в виде матрицы}





