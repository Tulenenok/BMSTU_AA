\chapter{Технологическая часть}

В данном разделе были приведены требования к программному обеспечению, средства реализации и листинги кода.

\section{Требования к программному обеспечению}

Программа принимает на вход две матрицы A и B. Количество столбцов матрицы A должно быть равно количеству строк матрицы B. На выходе получается результат умножения матриц, введенных пользователем.


\section{Средства реализации}

В качестве языка программирования для реализации данной лабораторной работы был выбран язык программирования Python \cite{pythonlang}. 

Данный язык имеет все небходимые инструменты для решения постав-
ленной задачи.

Время работы алгоритмов было замерено с помощью функции time() из библиотеки time \cite{pythonlangtime}

\section{Сведения о модулях программы}
Программа состоит из четырех модулей:
\begin{enumerate}
	\item algorithms.py - хранит реализацию алгоритмов сортировок;
	\item unit\_tests.py - хранит реализацию тестирующей системы и тесты;
	\item time\_memory.py - хранит реализацию системы замера памяти и времени;
	\item tools.py - хранит реализацию вспомогательных функций.
\end{enumerate}


\section{Реализация алгоритмов}

В листингах \ref{lst:standart}, \ref{lst:vinograd}, \ref{lst:vinograd_opt} представлены реализации алгоритмов умножения матриц - стандартного, Винограда и оптимизированного алгоритма Винограда.


\begin{lstlisting}[label=lst:standart,caption=Реализация стандарного умножения матриц]
def simple_matrix_mult(m1, m2):
    if len(m1[0]) != len(m2):
        print("Error")
        return -1

    n = len(m1)
    m = len(m1[0])
    q = len(m2[0])

    m_res = [[0] * q for _ in range(n)]

    for i in range(n):
        for j in range(q):
            for k in range(m):
                m_res[i][j] = m_res[i][j] + m1[i][k] * m2[k][j]

    return m_res
\end{lstlisting}

\begin{lstlisting}[label=lst:vinograd,caption=Реализация алгоритма Копперсмита-Винограда]
def winograd_matrix_mult(matrix1, matrix2):
	if (len(matrix2) != len(matrix1[0])):
		print("Error")
		return -1
	
	n = len(matrix1)
	m = len(matrix1[0])
	q = len(matrix2[0])
	
	matrix_res = [[0] * q for i in range(n)]
	
	row_factor = [0] * n
	for i in range(n):
		for j in range(0, m // 2, 1):
			row_factor[i] = row_factor[i] + \
				matrix1[i][2 * j] * matrix1[i][2 * j + 1]
	
	column_factor = [0] * q
	for i in range(q):
		for j in range(0, m // 2, 1):
			column_factor[i] = column_factor[i] + \
				matrix2[2 * j][i] * matrix2[2 * j + 1][i]
	
	for i in range(n):
		for j in range(q):
			matrix_res[i][j] = -row_factor[i] - column_factor[j]
			for k in range(0, m // 2, 1):
				matrix_res[i][j] = matrix_res[i][j] +\
					(matrix1[i][2 * k + 1] + matrix2[2 * k][j]) * \
					(matrix1[i][2 * k] + matrix2[2 * k + 1][j])
	
	if m % 2 == 1:
		for i in range(n):
			for j in range(q):
				matrix_res[i][j] = matrix_res[i][j] +\
					matrix1[i][m - 1] * matrix2[m - 1][j]
	
	return matrix_res
\end{lstlisting}

\begin{lstlisting}[label=lst:vinograd_opt,caption=Реализация алгоритма Копперсмита-Винограда (оптимизированный)]
def winograd_matrix_mult_opim(matrix1, matrix2):
	if (len(matrix2) != len(matrix1[0])):
		print("Error")
		return -1
	
	n = len(matrix1)
	m = len(matrix1[0])
	q = len(matrix2[0])
	
	matrix_res = [[0] * q for i in range(n)]
	
	row_factor = [0] * n
	for i in range(n):
		for j in range(1, m, 2):
			row_factor[i] += matrix1[i][j] * matrix1[i][j - 1]
	
	column_factor = [0] * q
	for i in range(q):
		for j in range(1, m, 2):
			column_factor[i] += matrix2[j][i] * matrix2[j - 1][i]
	
	flag = n % 2
	for i in range(n):
		for j in range(q):
			matrix_res[i][j] = -(row_factor[i] + column_factor[j])
			for k in range(1, m, 2):
				matrix_res[i][j] += (matrix1[i][k - 1] + matrix2[k][j]) * \
						(matrix1[i][k] + matrix2[k - 1][j])
			if (flag):
				matrix_res[i][j] += matrix1[i][m - 1] \
					* matrix2[m - 1][j]

	return matrix_res
\end{lstlisting}


\section{Функциональные тесты}

В таблице~\ref{tbl:functional_test} приведены тесты для функций, реализующих стандартный алгоритм умножения матриц, алгоритм Винограда и оптимизированный алгоритм Винограда. Тесты пройдены успешно.

\clearpage

\begin{table}[h]
	\begin{center}
		\begin{threeparttable}
		\captionsetup{justification=raggedright,singlelinecheck=off}
		\caption{\label{tbl:functional_test} Функциональные тесты}
		\begin{tabular}{|c@{\hspace{7mm}}|c@{\hspace{7mm}}|c@{\hspace{7mm}}|c@{\hspace{7mm}}|c@{\hspace{7mm}}|c@{\hspace{7mm}}|}
			\hline
			Матрица А & Матрица В & Ожидаемый результат \\ 
			\hline

			$\begin{pmatrix}
				1 & 1 & 1
			\end{pmatrix}$ &
			$\begin{pmatrix}
				1 \\
				1 \\
				1 
			\end{pmatrix}$ &
			$\begin{pmatrix}
				1 & 1 & 1\\
				1 & 1 & 1 \\
				1 & 1 & 1
			\end{pmatrix}$ \\
                
                \hline

			$\begin{pmatrix}
				    5 & -1
			\end{pmatrix}$ &
			$\begin{pmatrix}
				-1 & 10
			\end{pmatrix}$ &
			Неверный размер\\
   
                \hline

			$\begin{pmatrix}
				1 & 0 & 0\\
				1 & 1 & 0 \\
				    0 & 0 & 1
			\end{pmatrix}$ &
			$\begin{pmatrix}
				1 \\
				    4 \\
				    2
			\end{pmatrix}$ &
			$\begin{pmatrix}
				1 & 0 & 0\\
				0 & 4 & 0 \\
				    0 & 0 & 2
			\end{pmatrix}$ \\
   
                \hline

			$\begin{pmatrix}
				1 & 2\\
				2 & 1
			\end{pmatrix}$ &
			$\begin{pmatrix}
				1 & 1\\
				1 & 1
			\end{pmatrix}$ &
			$\begin{pmatrix}
				3 & 3\\
				3 & 3
			\end{pmatrix}$ \\
   
                \hline

			$\begin{pmatrix}
				1
			\end{pmatrix}$ &
			$\begin{pmatrix}
				3
			\end{pmatrix}$ &
			$\begin{pmatrix}
				3
			\end{pmatrix}$ \\
    
                \hline

                $\begin{pmatrix}
				1 & 2 & 3\\
				4 & 5 & 6\\
				    1 & 2 & 3
			\end{pmatrix}$ &
			$\begin{pmatrix}
				1 & 2 & 3\\
				4 & 5 & 6\\
				    1 & 2 & 3
			\end{pmatrix}$ &
			$\begin{pmatrix}
				12 & 18 & 24\\
				30 & 45 & 60\\
				12 & 18 & 24
			\end{pmatrix}$ \\

            \hline

		\end{tabular}
		\end{threeparttable}
	\end{center}
	
\end{table}


\section*{Вывод}

В этом разделе была представлена реализация алгоритмов классического умножения матриц, алгоритма Винограда, оптимизированного алгоритма Винограда. Тестирование показало, что алгоритмы реализованы
правильно и работают корректно.
