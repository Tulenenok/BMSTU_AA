\chapter*{Введение}
\addcontentsline{toc}{chapter}{Введение}

Матрицы используются во многих сферах деятельности. Например, ими пользуются при решении различных практических задач в математике, биологии, физике, технике, химии, экономике, маркетинге, психологии и других областях науки. 

Оптимизация операций работы над матрицами является важной задачей в программировании, так как размеры матриц могут достигать больших значений. Об оптимизации операции умножения пойдет речь в данной лабораторной работе.


\textbf{Целью данной работы} является изучение, реализация и исследование алгоритмов умножения матриц - классический алгоритм, алгоритм Винограда и оптимизированный алгоритм Винограда. 
Для достижения поставленной цели необходимо выполнить следующие задачи:
\begin{itemize}
	\item изучить и реализовать алгоритмы - классический, Винограда и его оптимизацию;
    \item провести тестирование по времени и по памяти для алгоритмов лабораторной работы;
    \item провести сравнительный анализ по времени классического алгоритма и алгоритма Винограда;
    \item провести сравнительный анализ по времени алгоритма Винограда и его оптимизации;
	\item описать и обосновать полученные результаты в отчете о выполненной лабораторной работе, выполненного как расчётно-пояснительная записка к работе.
\end{itemize}
