\chapter{Аналитическая часть}
В этом разделе было рассмотрено понятие многопоточности и представлено описание алгоритма Z-буфера.

\section{Многопоточность}

\textbf{Многопоточность} \cite{threads} -- способность центрального процессора или одного ядра в многоядерном процессоре одновременно выполнять несколько процессов или потоков, соответствующим образом поддерживаемых операционной системой.

Процесс -- это программа в ходе своего выполнения. Когда мы выполняем программу или приложение, запускается процесс. Каждый процесс состоит из одного или нескольких потоков.

Поток -- это сегмент процесса. Потоки представляют собой исполняемые сущности, которые выполняют задачи, стоящие перед исполняемым приложением. Процесс завершается, когда все потоки заканчивают выполнение.

Каждый поток в процессе -- это задача, которую должен выполнить процессор. Большинство процессоров сегодня умеют выполнять одновременно две задачи на одном ядре, создавая дополнительное виртуальное ядро. Это называется одновременная многопоточность или многопоточность $Hyper-Threading$, если речь о процессоре от Intel. 

Эти процессоры называются многоядерными процессорами. Таким образом, двухъядерный процессор имеет 4 ядра: два физических и два виртуальных. Каждое ядро может одновременно выполнять только один поток.

Как упоминалось выше, один процесс содержит несколько потоков, и одно ядро процессора может выполнять только один поток за единицу времени. Если мы пишем программу, которая запускает потоки последовательно, то есть передает выполнение в очередь одного конкретного ядра процессора, мы не раскрываем весь потенциал многоядерности. Остальные ядра просто стоят без дела, в то время как существуют задачи, которые необходимо выполнить. Если мы напишем программу таким образом, что она создаст несколько потоков для отнимающих много времени независимых функций, то мы сможем использовать другие ядра процессора, которые в противном случае пылились бы без дела. Можно выполнять эти потоки параллельно, тем самым сократив общее время выполнения процесса.


\section{Алгоритм Z-буфера}

Суть данного алгоритма -- это использование двух буферов: буфера
кадра, в котором хранятся атрибуты каждого пикселя, и Z-буфера, в котором
хранятся информация о координате Z для каждого пикселя.

Первоначально в Z-буфере находятся минимально возможные значения
Z, а в буфере кадра располагаются пиксели, описывающие фон. Каждый
многоугольник преобразуется в растровую форму и записывается в буфер
кадра.

В процессе подсчета глубины нового пикселя, он сравнивается с тем
значением, которое уже лежит в Z-буфере. Если новый пиксель расположен
ближе к наблюдателю, чем предыдущий, то он заносится в буфер кадра и
происходит корректировка Z-буфера.

Для решения задачи вычисления глубины Z каждый многоугольник
описывается уравнением $ax + by + cz + d = 0$. При $c = 0$, многоугольник для
наблюдателя вырождается в линию.

Преимуществами данного алгоритма являются простота реализации, а также линейная оценка трудоемкости.

Недостатки алгоритма - большой объем требуемой памяти и сложная реализация прозрачности.

\section*{Вывод}

В данном разделе был рассмотрен алгоритм удаления невидимых граней, использующий Z-буфер, а также представлена информация по поводу многопоточности.
