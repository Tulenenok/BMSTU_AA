\chapter{Исследовательская часть}

В данном разделе были приведены примеры работы программ, постановка эксперимента и сравнительный анализ алгоритмов на основе полученных данных.

\section{Технические характеристики}

Технические характеристики устройства, на котором выполнялось тестирование:

\begin{itemize}
	\item Операционная система macOS Monterey 12.5.1 
	\item Память 16 Гб.
	\item Процессор 2,3 ГГц 4‑ядерный процессор Intel Core i5.
\end{itemize}

Во время тестирования устройство было подключено к сети электропитания, нагружено приложениями окружения и самой  системой тестирования.

\section{Время выполнения алгоритмов}

Для \hfill замера \hfill процессорного \hfill времени \hfill использовалась \hfill функция 
\\ \textit{std::chrono::system\_clock::now(...)} из библиотеки $chrono$ \cite{cpp-lang-chrono} на C++. Функция возвращает процессорное время типа float в секундах.

Контрольная точка возвращаемого значения не определена, поэтому допустима только разница между результатами последовательных вызовов.

Замеры времени для каждой длины входного массива полигонов проводились 1000 раз. В качестве результата взято среднее время работы алгоритма на данной длине. При каждом запуске алгоритма, на вход подавались случайно сгенерированные массивы полигонов. 

\subsection*{Сравнение времени выполнения реализаций алгоритмов}

Результаты замеров приведены в таблице \ref{tbl:wor}.

\captionsetup{justification=raggedright,singlelinecheck=false}
\begin{table}[h]
	\begin{center}
		\caption{\label{tbl:wor} Время выполнения программ, реализующих последовательный и конвейерный алгоритм удаления невидимых граней, использующий Z-буфер в микросекундах.}
		\begin{tabular}{|c|r|r|}
			\hline				
			\multirow{3}{*}{\makecell{Количество\\ треугольников\\ }} & 	\multicolumn{2}{c|}{Количество потоков} \\ [3ex]
			\cline{2-3}	&Последовательный&	Конвейерный\\
			\hline		
			10&	    80787&	195937\\
			\hline		
			100&	126215&	541271\\
			\hline		
			1000&	224339&	662793\\
			\hline		
			10000&	1544327&	868520\\
			\hline		
			100000&	45514333&	12008429\\
			\hline			
			
		\end{tabular}
	\end{center}
\end{table}

\img{100mm}{trian}{Сравнение времени выполнения различных реализаций}


Программа, реализующая конвейерный подход к обработке данных,
выполняется медленнее программы, работающей последовательно, при относительно малых количествах треугольников (не более 10000). С увеличением количества треугольников все больше проявляется преимущество в
скорости конвейерного подхода, это происходит потому, что с ростом объема входных данных, время, расходуемое на диспетчерезацию конвейера,
становится пренебрежимо мало, в сравнении с временем обработки данных.

\section*{Вывод}

В данном разделе было произведено сравнение последовательной реализации трех алгоритмов и конвейера с использованием многопоточности. По
результатам исследования можно сказать, что конвейерную обработку выгоднее применять на больших числах (большие длины массивов, большое
количество задач), так как на малых размерах последовательный алгоритм
выигрывает у конвейерного.


