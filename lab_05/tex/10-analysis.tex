\chapter{Аналитическая часть}
В этом разделе было рассмотрено понятие конвейера и представлено описание алгоритма Z-буфера.

\section{Конвейерная обработка данных}

Конвейер — способ организации вычислений, используемый в современных процессорах и контроллерах с целью повышения их производительности (увеличения числа инструкций, выполняемых в единицу времени — эксплуатация параллелизма на уровне инструкций), технология, используемая при разработке компьютеров и других цифровых электронных
устройств.

Конвейерную обработку можно использовать для совмещения этапов выполнения разных команд. Производительность при этом возрастает благодаря тому, что одновременно на различных ступенях конвейера выполняются несколько команд ~\cite{second_article}.



\section{Алгоритм Z-буфера}

Суть данного алгоритма -- это использование двух буферов: буфера
кадра, в котором хранятся атрибуты каждого пикселя, и Z-буфера, в котором
хранятся информация о координате Z для каждого пикселя.

Первоначально в Z-буфере находятся минимально возможные значения
Z, а в буфере кадра располагаются пиксели, описывающие фон. Каждый
многоугольник преобразуется в растровую форму и записывается в буфер
кадра.

В процессе подсчета глубины нового пикселя, он сравнивается с тем
значением, которое уже лежит в Z-буфере. Если новый пиксель расположен
ближе к наблюдателю, чем предыдущий, то он заносится в буфер кадра и
происходит корректировка Z-буфера.

Для решения задачи вычисления глубины Z каждый многоугольник
описывается уравнением $ax + by + cz + d = 0$. При $c = 0$, многоугольник для
наблюдателя вырождается в линию.

Преимуществами данного алгоритма являются простота реализации, а также линейная оценка трудоемкости.

Недостатки алгоритма - большой объем требуемой памяти и сложная реализация прозрачности.

\section*{Вывод}

В данном разделе был рассмотрен алгоритм удаления невидимых граней, использующий Z-буфер, а также основные идеи конвейерной обработки данных.
