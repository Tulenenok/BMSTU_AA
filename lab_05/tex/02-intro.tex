\chapter*{Введение}
\addcontentsline{toc}{chapter}{Введение}

Разработчики архитектуры компьютеров издавна прибегали к методам проектирования, известным под общим названием "совмещение операций", при котором аппаратура компьютера в любой момент времени выполняет одновременно более одной базовой операции. 

Этот общий метод включает в себя, в частности, такое понятие, как конвейеризация. Конвейры широко применяются программистами для решения трудоемких задач, которые можно разделить на этапы, а также в большинстве современных быстродействующих процессоров ~\cite{first_article}. 


Целью данной работы является изучение организации конвейерной обработки данных на базе алгоритма стандартизации массива.


В рамках выполнения работы необходимо решить следующие задачи: 
\begin{enumerate}[label={\arabic*)}]
	\item изучить основы конвейеризации;
	\item изучить алгоритм Z-буфера;
	\item разработать последовательную реализацию алгоритма Z-буфера;
	\item разработать конвейерную реализацию алгоритма Z-буфера;
	\item провести сравнительный анализ времени работы реализаций.
\end{enumerate}