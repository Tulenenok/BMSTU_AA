\chapter{Конструкторская часть}
В этом разделе были представлены требования к вводу и программе, а также схемы алгоритма полного перебора и муравьиного алгоритма.

\section{Требования к вводу}
На вход программе должна подаваться матрица стоимостей, которая задает взвешенный неориентированный граф. 

\section{Требования к программе}
Выходные данные программы -- оптимальный маршрут, проходящий через все заданные вершины по одному разу с последующим возвратом в исходную точку, и его стоимость. Программа должна работать в рамках следующих ограничений: 

\begin{itemize}
 \item стоимости путей должны быть целыми числами;
 \item число городов должно быть больше 1;
 \item число дней должно быть больше 0;
 \item параметры муравьиного алгоритма должны быть вещественными числами, большими 0;
 \item матрица должна задавать неориентированный граф;
 \item должно быть выдано сообщение об ошибке при некорректном вводе параметров.
\end{itemize}

Пользователь должен иметь возможность выбора метода решения - полным перебором или муравьиным алгоритмом, и вывода результата на экран. Кроме того должна быть возможность проведения параметризации муравьиного алгоритма. Также должны быть реализованы сравнение алгоритмов по времени работы с выводом результатов на экран и получение графического представления результатов сравнения. Данные действия пользователь должен выполнять при помощи меню.

\section{Разработка алгоритмов}

На рисунке \ref{img:full_comb} представлена схема алгоритма полного перебора путей, а на рисунках \ref{img:ant_alg_part1}--\ref{img:ant_alg_part2} схема муравьиного алгоритма поиска путей. Также на рисунках \ref{img:find_pos}--\ref{img:update_phero} представлены схемы вспомогательных функций для муравьиного алгоритма.

\img{250mm}{full_comb}{Схема алгоритма полного перебора путей}
\img{250mm}{ant_alg_part1}{Схема муравьиного алгоритма (часть 1)}
\img{245mm}{ant_alg_part2}{Схема муравьиного алгоритма (часть 2)}
\img{220mm}{find_pos}{Схема алгоритма нахождения массива вероятностных переходов в непосещенные города}
\img{200mm}{rand_choice}{Схема алгоритма нахождения следующего города на основании рандома}
\img{250mm}{update_phero}{Схема алгоритма обновления матрицы феромонов}

\clearpage

\section*{Вывод}

В данном разделе были построены схемы алгоритмов, рассматриваемых в лабораторной работе, были описаны требования ко вводу и программе.