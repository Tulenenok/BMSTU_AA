\chapter*{Введение}
\addcontentsline{toc}{chapter}{Введение}

Современные средства навигации, организация логистики, конвейерного производства, анализ эффективности финансовых инструментов строятся на алгоритмах решения задачи поиска оптимального решения по выбранному параметру в сложной системе.
Данную задачу высокой вычислительной сложности называют задачей коммивояжера \cite{task}.

Задачи высокой вычислительной сложности могут быть решены при помощи полного перебора вариантов и эвристических алгоритмов \cite{evr}. Смысл понятия ''эвристический алгоритм'' состоит в том, что в этом случае алгоритм не вытекает из строгих положений теории, а в значительной степени основан на интуиции и опыте. Такие методы могут давать удовлетворительные результаты при вероятностных параметрах. Алгоритмы, основанные на использовании эвристических алгоритмов, не всегда приводят к оптимальным решениям. Однако для их применения на практике достаточно, чтобы ошибка прогнозирования не превышала допустимого значения, а этого можно добиться, например, подбором более информативных параметров.

Целью данной лабораторной работы является реализация муравьиного алгоритма и приобретение навыков параметризации методов на примере реализованного алгоритма, примененного к задаче коммивояжера.

Для достижения данной цели необходимо решить следующие задачи.

\begin{enumerate}
	\item Изучить алгоритм полного перебора для решения задачи коммивояжера.
	\item Реализовать алгоритм полного перебора для решения задачи коммивояжера.
	\item Изучить муравьиный алгоритм для решения задачи коммивояжера.
	\item Реализовать муравьиный алгоритм для решения задачи коммивояжера.
	\item Провести параметризацию муравьиного алгоритма на трех классах
	данных.
	\item Провести сравнительный анализ скорости работы реализованных
	алгоритмов.
	\item Подготовить отчет о выполненной лабораторной работе.
\end{enumerate}