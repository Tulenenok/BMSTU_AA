\chapter{Исследовательская часть}

В данном разделе были произведен сравнителный анализ алгоритмов.

\section{Технические характеристики}

Технические характеристики устройства, на котором выполнялось тестирование:

\begin{itemize}
	\item Операционная система macOS Monterey 12.5.1 
	\item Память 16 Гб.
	\item Процессор 2,3 ГГц 4‑ядерный процессор Intel Core i5.
\end{itemize}

Во время тестирования устройство было подключено к сети электропитания, нагружено приложениями окружения и самой  системой тестирования.

\section{Время выполнения алгоритмов}

Для \hfill замера \hfill процессорного \hfill времени \hfill использовалась \hfill функция 
\\ \textit{std::chrono::system\_clock::now(...)} из библиотеки $chrono$ \cite{cpp-lang-chrono} на C++. Функция возвращает процессорное время типа float в секундах.

Контрольная точка возвращаемого значения не определена, поэтому допустима только разница между результатами последовательных вызовов.

Замеры времени для каждого размера матрицы проводились 500 раз. В качестве результата взято среднее время работы алгоритма на данном размере. При каждом запуске алгоритма, на вход подавались случайно сгенерированные матриц. Тестовые пакеты создавались до начала замера времени.

Результаты замеров приведены в таблице \ref{tbl:wor}.

\begin{table}[h]
    \begin{center}
    \captionsetup{justification=raggedright,singlelinecheck=off}
    \caption{\label{tbl:wor} Результаты замеров времени}
        \begin{tabular}{|r|r|r|}
            \hline Размер & Полный перебор & Муравьиный алгоритм \\ \hline
            2 &   0.000130 &   0.019932 \\ \hline
            3 &   0.000138 &   0.031615 \\ \hline
            4 &   0.000104 &   0.044361 \\ \hline
            5 &   0.000420 &   0.089291 \\ \hline
            6 &   0.002390 &   0.152131 \\ \hline
            7 &   0.019703 &   0.254059 \\ \hline
            8 &   0.162850 &   0.398472 \\ \hline
            9 &   1.637611 &   0.594024 \\ \hline
            10 &  18.207853 &  0.857666  \\ \hline
        \end{tabular}
    \end{center}
\end{table}


\img{110mm}{trian}{Сравнение по времени алгоритмов полного перебора путей и муравьиного на разных размерах матриц}


\section{Параметризация}

Целью проведения параметризации является определение таких комбинаций параметров, при которых муравьиный алгоритм даёт наилучшие результаты.

В результате автоматической параметризации будет получена таблицы со следующими столбцами:
\begin{itemize}
	\item коэффициент видимости $\alpha$ - изменяющийся параметр;
	\item коэффициент испарения феромона $\rho$ - изменяющийся параметр;
	\item число дней $days$ - изменяющийся параметр;
	\item эталонный результат $ideal$;
	\item разность полученного при данных параметрах значения и эталонного $mistake$.
\end{itemize}

\subsection{Класс данных 1}

Класс данных 1 представляет собой матрицу стоимостей в диапозоне от 1 до 5 для 8 городов:

\begin{equation}
    \label{eq:kd1}
	K_{1} = \begin{pmatrix}
		0 & 4 & 3 & 2 & 4 & 5 & 5 & 3 \\
		4 & 0 & 4 & 5 & 5 & 1 & 4 & 5 \\
		3 & 4 & 0 & 5 & 4 & 1 & 2 & 1 \\
		2 & 5 & 5 & 0 & 1 & 3 & 1 & 5 \\
		4 & 5 & 4 & 1 & 0 & 2 & 5 & 4 \\
		5 & 1 & 1 & 3 & 2 & 0 & 4 & 5 \\
		5 & 4 & 2 & 1 & 5 & 4 & 0 & 2 \\
		3 & 5 & 1 & 5 & 4 & 5 & 2 & 0 \\
	\end{pmatrix}
\end{equation}

Для данного класса данных приведена таблица с выборкой параметров, которые наилучшим образом решают поставленную задачу.

\begin{center}
    \captionsetup{justification=raggedright,singlelinecheck=off}
    \begin{longtable}[c]{|c|c|c|c|c|}
    \caption{Параметры для класса данных 1\label{tbl:table_kd1}}\\ \hline
        $\alpha$ & $\rho$ & Days & Result & Mistake \\ \hline
 0.1 &  0.9 &  100 &    15 &     0 \\
 0.1 &  0.9 &  200 &    15 &     0 \\
 0.1 &  0.9 &  300 &    15 &     0 \\
 0.1 &  0.9 &  400 &    15 &     0 \\
 0.1 &  0.9 &  500 &    15 &     0 \\
\hline
 0.2 &  0.8 &  100 &    15 &     0 \\
 0.2 &  0.8 &  200 &    15 &     0 \\
 0.2 &  0.8 &  300 &    15 &     0 \\
 0.2 &  0.8 &  400 &    15 &     0 \\
 0.2 &  0.8 &  500 &    15 &     0 \\
\hline
 0.3 &  0.7 &  100 &    15 &     0 \\
 0.3 &  0.7 &  200 &    15 &     0 \\
 0.3 &  0.7 &  300 &    15 &     0 \\
 0.3 &  0.7 &  400 &    15 &     0 \\
 0.3 &  0.7 &  500 &    15 &     0 \\
\hline
 0.4 &  0.6 &  100 &    15 &     0 \\
 0.4 &  0.6 &  200 &    15 &     0 \\
 0.4 &  0.6 &  300 &    15 &     0 \\
 0.4 &  0.6 &  400 &    15 &     0 \\
 0.4 &  0.6 &  500 &    15 &     0 \\
\hline
 0.5 &  0.5 &  100 &    15 &     0 \\
 0.5 &  0.5 &  200 &    15 &     0 \\
 0.5 &  0.5 &  300 &    15 &     0 \\
 0.5 &  0.5 &  400 &    15 &     0 \\
 0.5 &  0.5 &  500 &    15 &     0 \\
\hline
 0.6 &  0.4 &  100 &    15 &     0 \\
 0.6 &  0.4 &  200 &    15 &     0 \\
 0.6 &  0.4 &  300 &    15 &     0 \\
 0.6 &  0.4 &  400 &    15 &     0 \\
 0.6 &  0.4 &  500 &    15 &     0 \\
\hline
\endfirsthead
\captionsetup{labelformat=continued, labelsep=quad}%
\caption{\space}\\
\endhead
 0.7 &  0.3 &  100 &    15 &     0 \\
 0.7 &  0.3 &  200 &    15 &     0 \\
 0.7 &  0.3 &  300 &    15 &     0 \\
 0.7 &  0.3 &  400 &    15 &     0 \\
 0.7 &  0.3 &  500 &    15 &     0 \\
\hline
 0.8 &  0.2 &  100 &    15 &     0 \\
 0.8 &  0.2 &  200 &    15 &     0 \\
 0.8 &  0.2 &  300 &    15 &     0 \\
 0.8 &  0.2 &  400 &    15 &     0 \\
 0.8 &  0.2 &  500 &    15 &     0 \\
\hline
 0.9 &  0.1 &  100 &    15 &     0 \\
 0.9 &  0.1 &  200 &    15 &     0 \\
 0.9 &  0.1 &  300 &    15 &     0 \\
 0.9 &  0.1 &  400 &    15 &     0 \\
 0.9 &  0.1 &  500 &    15 &     0 \\
\hline
\end{longtable}
\end{center}

\subsection{Класс данных 2}

Класс данных 2 представляет собой матрицу стоимостей в диапозоне от 5000 до 9000 для 8 городов:

\begin{equation}
    \label{eq:kd2}
	K_{1} = \begin{pmatrix}
		0 & 5466 & 8308 & 8068 & 7284 & 5635 & 6055 & 8129 \\
		5466 & 0 & 8205 & 7384 & 6794 & 6048 & 6174 & 6306 \\
		8308 & 8205 & 0 & 5485 & 7872 & 7981 & 7868 & 6912 \\
		8068 & 7384 & 5485 & 0 & 7002 & 6683 & 7544 & 8278 \\
		7284 & 6794 & 7872 & 7002 & 0 & 5159 & 8240 & 5663 \\
		5635 & 6048 & 7981 & 6683 & 5159 & 0 & 8801 & 8844 \\
		6055 & 6174 & 7868 & 7544 & 8240 & 8801 & 0 & 5493 \\
		8129 & 6306 & 6912 & 8278 & 5663 & 8844 & 5493 & 0 \\
	\end{pmatrix}
\end{equation}

Для данного класса данных приведена таблица с выборкой параметров, которые наилучшим образом решают поставленную задачу.

\begin{center}
    \captionsetup{justification=raggedright,singlelinecheck=off}
    \begin{longtable}[c]{|c|c|c|c|c|}
    \caption{Параметры для класса данных 2\label{tbl:table_kd2}}\\ \hline
        $\alpha$ & $\rho$ & days & ideal & mistake \\ \hline
    
 0.1 &  0.7 &  300 & 47326 &     0 \\
 0.1 &  0.7 &  400 & 47326 &     0 \\
 0.1 &  0.7 &  500 & 47326 &     0 \\
 \hline
 0.2 &  0.5 &  300 & 47326 &     0 \\
 0.2 &  0.5 &  400 & 47326 &     0 \\
 0.2 &  0.5 &  500 & 47326 &     0 \\
 \hline
 0.3 &  0.3 &  300 & 47326 &     0 \\
 0.3 &  0.3 &  400 & 47326 &     0 \\
 0.3 &  0.3 &  500 & 47326 &     0 \\
\hline
 0.4 &  0.1 &  300 & 47326 &     0 \\
 0.4 &  0.1 &  400 & 47326 &     0 \\
 0.4 &  0.1 &  500 & 47326 &     0 \\
\hline
 0.5 &  0.6 &  300 & 47326 &     0 \\
 0.5 &  0.6 &  400 & 47326 &     0 \\
 0.5 &  0.6 &  500 & 47326 &     0 \\
 \hline
 0.6 &  0.8 &  300 & 47326 &     0 \\
 0.6 &  0.8 &  400 & 47326 &     0 \\
 0.6 &  0.8 &  500 & 47326 &     0 \\
\hline
 0.7 &  0.2 &  300 & 47326 &     0 \\
 0.7 &  0.2 &  400 & 47326 &     0 \\
 0.7 &  0.2 &  500 & 47326 &     0 \\
\hline
 0.8 &  0.6 &  300 & 47326 &     0 \\
 0.8 &  0.6 &  400 & 47326 &     0 \\
 0.8 &  0.6 &  500 & 47326 &     0 \\
\hline
 0.9 &  0.9 &  300 & 47326 &     0 \\
 0.9 &  0.9 &  400 & 47326 &     0 \\
 0.9 &  0.9 &  500 & 47326 &     0 \\
\hline
\end{longtable}
\end{center}

\section*{Вывод}

В результате эксперимента было получено, что для 2 городов полный перебор работает быстрее муравьиного алгоритма в 32 раза. Начиная с 8 городов, муравьиный алгоритм работает быстрее полного перебора: в 75 раз быстрее для 10 городов. Таким образом, муравьиный алгоритм необходимо использовать при большом числе городов - от 8 и более.

Также в результате проведения параметризации было установлено, что для первого класса данных лучшие результаты муравьиный алгоритм дает на следующих значениях параметров:
\begin{itemize}
	\item $\alpha$ = 0.1, $\rho$ = 0.1-0.5, 0.7-0.9;
	\item $\alpha$ = 0.2, $\rho$ = 0.1-0.7, 0.9;
	\item $\alpha$ = 0.3, $\rho$ = 0.2-0.6, 0.9;
	\item $\alpha$ = 0.4, $\rho$ = 0.5-0.9;
	\item $\alpha$ = 0.5, $\rho$ = 0.1, 0.5-0.9;
	\item $\alpha$ = 0.7, $\rho$ = 0.4-0.8.
\end{itemize}

Можно сделать вывод о том, что данные значения параметров следует использовать для матриц стоимостей в диапазоне от 1 до 5.

Для второго класса данных лучшие результаты муравьиный алгоритм дает на следующих значениях параметров:
\begin{itemize}
	\item $\alpha$ = 0.1, $\rho$ = 0.1, 0.4, 0.7;
	\item $\alpha$ = 0.2, $\rho$ = 0.3, 0.7, 0.9;
	\item $\alpha$ = 0.3, $\rho$ = 0.2, 0.6, 0.9;
	\item $\alpha$ = 0.4, $\rho$ = 0.7, 0.8;
	\item $\alpha$ = 0.5, $\rho$ = 0.2, 0.6-0.9;
	\item $\alpha$ = 0.5, $\rho$ = 0.3, 0.7-0.9;
	\item $\alpha$ = 0.6, $\rho$ = 0.2, 0.6-0.9;
	\item $\alpha$ = 0.8, $\rho$ = 0.1, 0.6;
	\item $\alpha$ = 0.9, $\rho$ = 0.9.
\end{itemize}

Можно сделать вывод о том, что данные значения параметров следует использовать для матриц стоимостей в диапазоне от 5000 до 9000.

Из результатов параметризации видно, что погрешность результата уменьшается при большом числе дней и меньшем значении коэффициента видимости.