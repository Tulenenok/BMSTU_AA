\chapter{Аналитическая часть}
В этом разделе будут представлены описания алгоритмов гномьей сортировки, плавной сортировки и сортировки двоичным деревом поиска.

\section{Гномья сортировка}

\textbf{Гномья сортировка \cite{Knut}} --  алгоритм сортировки, который использует только один цикл, что является редкостью. В этой сортировке массив просматривается слева-направо, при этом сравниваются и, если нужно, меняются соседние элементы. Если происходит обмен элементов, то происходит возвращение на один шаг назад. Если обмена не было - алгоритм продолжает просмотр массива в поисках неупорядоченных пар.


\section{Плавная сортировка}

\textbf{Плавная сортировка \cite{Knut}} -- алгоритм сортировки выбором, разновидность пирамидальной сортировки. От классического алгоритма пирамидальной сортировки отличается тем, что его сложность зависит от степени изначальной упорядоченности входного массива, на основе которого строится "двоичная куча". Эта сортировка подразумевает два основных шага: формирование последовательности куч и получение на основе на этой последовательности отсортированного массива.

\section{Сортировка бинарным деревом}

\textbf{Сортировка бинарным деревом \cite{Knut}} -- универсальный алгоритм сортировки, заключающийся в построении двоичного дерева поиска по ключам массива, с последующей сборкой результирующего массива путём обхода узлов построенного дерева в необходимом порядке следования ключей.

\newpage
Шаги алгоритма:
\begin{enumerate}[label=\arabic*)]
	\item построить двоичное дерево поиска по ключам массива;
	\item собрать результирующий массив путём обхода узлов дерева поиска в необходимом порядке следования ключей;
	\item вернуть, в качестве результата, отсортированный массив.
\end{enumerate}

\section*{Вывод}

В данной работе необходимо реализовать алгоритмы сортировки, описанные в данном разделе, а также провести их теоритическую оценку и проверить ее экспериментально.



