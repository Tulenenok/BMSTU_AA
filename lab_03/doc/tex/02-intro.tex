\chapter*{Введение}
\addcontentsline{toc}{chapter}{Введение}

Сортировка -- процесс перегруппировки заданной последовательности объектов в определенном порядке. Это одна из главных процедур обработки структурированной информации.

Существует множество различных методов сортировки данных, однако любой алгоритм сортировки имеет:

\begin{itemize}
	\item сравнение, определяющее упорядочность пары элементов;
	\item перестановка, меняющая местами пару элементов;
	\item алгоритм сортировки, используюзщий сравнение и перестановки.
\end{itemize}

Алгоритмы сортировки имеют большое значение, так как позволяют эффективнее проводить работу с последовательностью данных. Например, возьмем задачу поиска элемента в последовательности -- при работе с отсортированным набором данных время, которое нужно на нахождение элемента, пропорционально логарифму количства элементов. Последовательность, данные которой расположены в хаотичном порядке, занимает время, которое пропорционально количеству элементов, что куда больше логарифма.

\bigskip

\textbf{Задачи данной лабораторной работы:}

\begin{enumerate}[label=\arabic*)]
	\item изучить и реализовать три алгоритма сортировки:
	\begin{enumerate}
		\item гномья сортировка; 
		\item плавная сортировка;
		\item сортировка бинарным деревом;
	\end{enumerate}
	\item провести сравнительный анализ трудоемкости алгоритмов на основе теоретических расчетов и выбранной модели вычислений;
	\item провести сравнительный анализ алгоритмов на основе экспериментальных данных по времени выполнения программы;
	\item описать и обосновать полученные результаты в отчете о выполненной лабораторной работе, выполненного как расчётно-пояснительная записка к работе.
\end{enumerate}
